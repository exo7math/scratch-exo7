\documentclass[class=report,crop=false, 12pt]{standalone}
\usepackage[screen]{../scratch}

\begin{document}

\titre[P]{Boucles II}
%===============================



\section*{Objectifs}

\begin{itemize}
  \item Élaboration de programmes sophistiqués.
  \item Boucle "pour".
\end{itemize}


\section*{Durée}

1 heure (??)

\section*{Les activités}

\begin{itemize}
  \item Voir les commentaires de "Boucles I".
  
  \item La boucle "pour $x$ dans l'ensemble $E$" est devenue un incontournable des langages modernes. On n'est plus obligé d'indexer les objets, puis de faire une boucle sur des entiers, on écrit directement "pour chaussette dans tiroir, affiche couleur(chaussette)". C'est beaucoup plus naturel. Scratch ne connaît pas vraiment cette boucle "pour", ni dans sa version numérique, ni dans sa version générale. Et c'est dommage ! 
  
\end{itemize}


\section*{Ressources}


\section*{People}

\begin{itemize}
  \item Auteur : Arnaud Bodin
\end{itemize}


\end{document}


