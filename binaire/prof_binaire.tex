\documentclass[class=report,crop=false, 12pt]{standalone}
\usepackage[screen]{../scratch}

\begin{document}

\titre[P]{Binaire}
%===============================



\section*{Objectifs}

\begin{itemize}
  \item Passage décimal vers binaire.
  \item Et dans l'autre sens.
\end{itemize}


\section*{Durée}

1 heure (??)

\section*{Les activités}

\begin{itemize}
  \item Le plus dur est d'abord de franchir le saut conceptuel : distinguer un nombre de son écriture : $16$, \textsc{xvi}, "seize" $10000_b$ représente le même objet, mais les écritures sont différentes.
  
  \item Commencer par revenir aux bases  de ce qu'est l'écriture décimale (centaines, dizaines, unités).
  
  \item Encore une fois l'écriture binaire avec des points (ex. $1.0.1.1$) n'est pas standard.
  
  \item Les divisions successives par $2$ ne devraient pas poser de problème. Les restes étant strictement inférieurs à $2$, c'est soit $0$ soit $1$. Il existe une autre disposition, plus compacte et en diagonale, pour les divisions successives (voir le lien plus bas).
  
  \item Possibilité de voir les nombres binaires comme l'énumération des entiers avec seulement des $0$ et $1$. Commencer par écrire en binaire les entiers $0$, $1$, $2$...
  Puis les faire écrire jusqu'à $20$ jusqu'à comprendre la logique (principe du compteur kilométrique d'une voiture qui bascule). On  peut ainsi chercher quel est le successeurs de $1.0.1.1.1$ sans savoir à quel entier cela correspond.
\end{itemize}


\section*{Ressources}

\begin{itemize}
  \item Lien vers une image avec divisions successives sous forme compacte : \href{http://www.apprendre-en-ligne.net/crypto/images/bases.html}{http://www.apprendre-en-ligne.net/crypto/images/bases.html}
\end{itemize}



\section*{People}

\begin{itemize}
  \item Auteur : Arnaud Bodin
\end{itemize}


\end{document}


