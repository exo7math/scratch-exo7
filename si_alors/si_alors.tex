\documentclass[class=report,crop=false, 12pt]{standalone}
\usepackage[screen]{../scratch}


\begin{document}

\titre[F]{Si ... alors ...}
%================================

Le test \emph{si ... alors ... sinon ...} permet d’exécuter des instructions différentes suivant la réalisation ou non d'une condition.
%selon que la condition est réalisée ou pas.

On schématise ce test par un diagramme avec un losange (à gauche) ; on peut aussi écrire les instructions ligne par ligne (à droite).

\begin{center}
\begin{minipage}{0.45\textwidth} 
\footnotesize\myfigure{0.85}{
\tikzinput{si_alors1}
}
\end{minipage}
\qquad
\qquad
\begin{minipage}{0.4\textwidth}
Si la condition est réalisée, alors :\\
\indentation on effectue des instructions \\
sinon :\\
\indentation on effectue d'autres instructions \\
\end{minipage}
\end{center}

Par exemple : voici des instructions qui, à partir des nombres $a$ et $b$, testent si $a$ est supérieur ou égal à $b$, et renvoient le plus grand. 

\begin{center}
\begin{minipage}{0.45\textwidth} 
\footnotesize\myfigure{0.75}{
\tikzinput{si_alors2}
}
\end{minipage}
\qquad
\qquad
\begin{minipage}{0.4\textwidth}
Demander $a$ et $b$ \\
Si $a \ge b$, alors :\\
\indentation renvoyer $a$ \\
sinon :\\
\indentation renvoyer $b$ \\
\end{minipage}
\end{center}



\begin{activite}
\sauteligne
\begin{enumerate}
  \item Comprends et explique ce que font les instructions suivantes.
\myfigure{0.9}{
\footnotesize\tikzinput{si_alors3}
}
  
  \item Comprends les instructions suivantes et dresse la table des valeurs renvoyées pour $x=1$, puis $x=2,3,\ldots, 10$.
  
\myfigure{0.9}{
\footnotesize\tikzinput{si_alors4}
}  
  
    $$\begin{array}{|l||c|c|c|c|c|c|c|c|c|c|c}
  \hline
  \text{entrée } x & 1&2&3&4&5&6&7&8&9&10 \\
  \hline
  \text{sortie } & &&&&&&&&& \\
  \hline
  \end{array}  
  $$ 
  
  
  \item Explique la réduction calculée par cet algorithme en fonction de l'âge.
  
\myfigure{0.8}{
\footnotesize\tikzinput{si_alors5}
} 



  \item Écris les instructions des questions précédentes sous la forme \og ligne par ligne \fg{}. 
  
  
\end{enumerate}
\end{activite}

\begin{activite}
Écris le diagramme des commandes qui permet de répondre aux problèmes suivants.
\begin{enumerate}
  \item On demande l'âge d'une personne. Soit elle est majeure et alors l'ordinateur répond \og Vous êtes majeur \fg{} ;
  soit il dit \og Vous serez majeur dans ... années \fg{}. 
  
  \item On demande deux durées de course d'une nageuse (en secondes). 
  \begin{itemize}
    \item L'ordinateur affiche sa meilleure performance ;
    
    \item si sa meilleure performance est inférieure ou égale à $100$, il affiche en plus \og Bravo, tu bats le record ! \fg{} ;
    
    \item sinon il affiche \og Tu es à ... secondes du record\fg{}.
  \end{itemize}
  
  Refais le même exercice avec trois durées.
  
  \item On demande un entier $x$, l'ordinateur renvoie un autre entier. Tu trouves ci-dessous les premiers exemples d'entrée/sortie de ce programme : 
%, sachant que les premières entrées/sorties sont les suivantes :
  $$\begin{array}{|l||c|c|c|c|c|c|c|c|c|c|c|c|c|c|}
  \hline
  \text{entrée } x & 1&2&3&4&5&6&7&8&9&10&11&12\\
  \hline
  \text{sortie } & 2&3&6&5&10&7&14&9&18&11&22&13\\
  \hline
  \end{array}  
  $$
  
\end{enumerate}
\end{activite}


\begin{activite}
\sauteligne
\begin{enumerate}
  \item  
  \begin{enumerate}
    \item On considère l'initialisation $x \leftarrow 7$, puis les instructions suivantes :

 
\indentation si $x \ge 10$ alors : \\ 
\indentation\indentation $x \leftarrow x - 3$ \\
\indentation sinon : \\
\indentation\indentation $x \leftarrow 2 \times x$
 
Combien vaut $x$ maintenant ?
  
    \item Reprends la même question en partant de $x \leftarrow 12$.
  
    \item Trouve deux valeurs initiales de $x$ qui donnent le même résultat final.
  
  \end{enumerate}  
  \item
  \begin{enumerate}

    \item On considère l'initialisation $x \leftarrow 7$, puis les instructions suivantes :
    
\indentation si $x$ est impair et $x \ge 10$ alors : \\ 
\indentation\indentation $x \leftarrow x + 4$ \\
\indentation si $x$ est impair et $x < 10$ alors : \\ 
\indentation\indentation $x \leftarrow x + 3$ \\
\indentation si $x$ est pair et $x \ge 10$ alors : \\ 
\indentation\indentation $x \leftarrow x + 2$ \\
\indentation si $x$ est pair et $x < 10$ alors : \\ 
\indentation\indentation $x \leftarrow x + 1$ 
 
Combien vaut $x$ maintenant ?

    \item Reprends la même question en partant de $x \leftarrow 12$.
  
    \item Trouve deux valeurs initiales de $x$ qui donnent le même résultat final.  
  
  \end{enumerate}   
\end{enumerate}
\end{activite}


\begin{activite}
Quelle sera la valeur de $x$ à la fin de chacune des instructions suivantes ? 

\begin{enumerate}
  \item ~ \\
  \begin{minipage}{0.3\textwidth}
$x \leftarrow 1$ \\
répéter $10$ fois : \\
\indentation $x \leftarrow x + 1$
\end{minipage}
  
  \item ~ \\
  \begin{minipage}{0.3\textwidth}
$x \leftarrow 1$ \\
répéter $10$ fois : \\
\indentation $x \leftarrow 2 \times x$
\end{minipage}

  \item ~ \\
  \begin{minipage}{0.3\textwidth}
$x \leftarrow 1$ \\
répéter $10$ fois : \\
\indentation si $x$ est pair alors : \\ 
\indentation\indentation $x \leftarrow x + 1$ \\
\indentation sinon : \\
\indentation\indentation $x \leftarrow x + 3$
\end{minipage}
  
\end{enumerate}



\end{activite}
\end{document}

