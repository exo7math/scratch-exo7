\documentclass[class=report,crop=false, 12pt]{standalone}
\usepackage[screen]{../scratch}

\begin{document}

\titre[P]{Graphes}
%===============================



\section*{Objectifs}

\begin{itemize}
  \item Manipuler des graphes.
\end{itemize}


\section*{Durée}

2 heures (??)

\section*{Les activités}

\begin{itemize}
  \item Ici les graphes sont des graphes du plan, donc on s'interdit des croisement d'arêtes. 
  
  \item L'introduction au graphe se fait par le théorème des $4$ couleurs. Le passage d'une carte à un graphe est une abstraction qui devrait sembler naturelle.  En plus le théorème des 4 couleurs nécessite pour sa démonstration l'usage de l'ordinateur, qui doit vérifier plus de $1000$ situations compliquées. 
  \item Les notions de plus court chemin, de chemin eulérien, de cycle eulérien sont abordés au travers d'exemples. 
  \item Pour tous ces thèmes il existe des algorithmes (exemple : algorithme de Dijkstra) qui sont peut-être abordables en collège, mais assez difficiles à formaliser (il faut les expliquer sur des exemples).
  
  \item La caractéristique d'Euler $\chi = S-A+F$ vaut ici $1$. On trouve plus fréquemment $\chi=2$. C'est dû au fait, qu'ici on ne compte pas la face non bornée.
  
  \item On pourra appliquer la formule $\chi=1$, au cas des triangulations, pour obtenir une relation entre le nombre de triangles, d'arêtes et de sommets. (Voir la feuille "Triangulation".)
\end{itemize}


\section*{Ressources}


\section*{People}

\begin{itemize}
  \item Auteur : Arnaud Bodin
\end{itemize}


\end{document}


