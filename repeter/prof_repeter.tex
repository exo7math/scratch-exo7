\documentclass[class=report,crop=false, 12pt]{standalone}
\usepackage{../scratch}

\begin{document}

\titre[P]{Répéter}
%===============================



\section*{Objectifs}

\begin{itemize}
  \item Premier type de boucle.
  \item Identifier des structures répétitives.
  \item Codage d'instructions.
\end{itemize}


\section*{Durée}

1 heure (??)

\section*{Les activités}

\begin{itemize}
  \item L'ordinateur prend tout son intérêt dans la répétition des tâches. On peut lui demander de répéter $1000$ fois la même chose, il ne se lasse pas.
  
  \item La boucle "répéter" est ici la plus basique : il s'agit de répéter un certain nombre de fois, déterminées à l'avance, une instruction ou un bloc d'instructions. 
  
  \item Identifier des motifs répétés permet aussi une écriture plus compacte et plus lisible : \mot{5N 3E} ou \mot{NNNNNEEE} ?
  
  \item Identifier les motifs répétés n'est pas si évident, et c'est le premier pas dans l'écriture d'une boucle. On ne cherche pas forcément une écriture unique des motifs.

\end{itemize}


\section*{Ressources}


\section*{People}

\begin{itemize}
  \item Auteur : Arnaud Bodin
\end{itemize}


\end{document}


