
\pagestyle{empty}\thispagestyle{empty}
\vspace*{\fill}
\begin{center}
\fontsize{52}{52}\selectfont
\textsc{algorithmes au collège}

  \hfil
  
\myfigure{0.8}{
  \tikzinput{voronoi}
}  
 % \includegraphics[width = 5cm]{Scratchcat}
\end{center}
\vfill
\begin{center}
\huge
\textsc{algorithmes \  et \  programmation}
\end{center}
\begin{center}
\LogoExoSept{2}
\end{center}
%\clearemptydoublepage
\clearpage

\thispagestyle{empty}

\vspace*{\fill}
\section*{À la découverte des algorithmes}

Un algorithme est une suite d'instructions données permettant d'atteindre un objectif ou de résoudre un problème, un peu comme une recette de cuisine. Comment effectuer une multiplication ? Comment trier une liste ? Quel est le plus court chemin entre deux villes ?

\bigskip

Un algorithme n'est pas lié à un langage, ni même aux ordinateurs ! C'est pourquoi on peut très bien comprendre un algorithme en travaillant sur feuille. Travailler sur feuille, pour faire de l'informatique, l'idée est surprenante. Mais ce travail permet d'abord de préparer ou de consolider les connaissances apprises devant la machine. Il permet également d'étudier des concepts difficiles à programmer avec un logiciel, comme par exemple des algorithmes graphiques ou bien encore qui portent sur les mots. 

\bigskip

Alan Turing est un personnage emblématique qui a été l'un des premiers à faire le lien entre travail théorique et travail sur machine. D'une part, il a participé activement à la création d'un des premiers ordinateurs permettant ainsi de décrypter des messages secrets durant la seconde guerre mondiale. D'autre part, il a conçu \emph{la machine de Turing}, encore utilisée de nos jours, qui n'est pas une véritable machine, mais un modèle d'ordinateur sur papier ! 

\vspace*{\fill}

%\newpage
\thispagestyle{empty}
\addtocontents{toc}{\protect\setcounter{tocdepth}{1}}
\tableofcontents
