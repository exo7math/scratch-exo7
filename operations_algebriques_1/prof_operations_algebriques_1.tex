\documentclass[class=report,crop=false, 12pt]{standalone}
\usepackage[screen]{../scratch}

\begin{document}

\titre[P]{Opérations algébriques I}
%===============================



\section*{Objectifs}

\begin{itemize}
  \item Structure d'arbre.
  \item Affectation, variable.
\end{itemize}


\section*{Durée}

2 heures (??)

\section*{Les activités}

\begin{itemize}
  \item Arbres pour les opérations algébriques.
  \begin{itemize}
    \item C'est une façon différente d'écrire une formule algébrique.
    \item L'écriture en arbre se rapproche de ce qui est véritablement stocké en interne dans une machine.
    \item Permet d'avoir un nouveau point de vue sur les opérations algébriques, l'ordre des priorités... 
    \item Permet de réviser la factorisation et le développement.
    \item On pourrait mettre en évidence la commutativité, l'associativité, la distributivité de certaines opérations (et pas de certaines autres).
  \end{itemize}
  
  \item Affectation. On a choisi la notation $x \leftarrow 2$. En mathématique l'écriture "$x=2$" est ambigüe. C'est soit "$x$ vaut $2$" (affectation) ou bien "est-ce que $x$ vaut $2$ ?" (une équation). En informatique, on est obligé de distinguer ces deux notions.

  \item Variable. Il est difficile de comprendre l'écriture $x \leftarrow x +3$. En mathématiques, on écrirait $x_0$, $x_1 = x_0+3$, $x_2 =x_1+3$... mais en informatique on utilise une seule variable pour économiser de la mémoire. L'écriture $x \leftarrow x +3$ demande un peu d'effort intellectuel, d'autant que cette expression se lit essentiellement de droite à gauche, contrairement à l'habitude.

\end{itemize}


\section*{Ressources}


\section*{People}

\begin{itemize}
  \item Auteur : Arnaud Bodin
\end{itemize}


\end{document}


