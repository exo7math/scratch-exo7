\documentclass[class=report,crop=false, 12pt]{standalone}
\usepackage[screen]{../scratch}


\begin{document}

% Commande spécifique
\newcommand{\hexa}{\text{hex}}


\titre[F]{Couleurs}
%===============================



\begin{activite}[Perception des couleurs]

%\sauteligne
%
%\begin{enumerate}
%\item \textbf{Spectre visible.} 


La lumière est une onde. La couleur de la lumière dépend de 
sa longueur d'onde. Les longueurs d'ondes visibles par l'\oe il humain  vont de $400$ à $700$ nanomètres environ (un nanomètre c'est $0,000 \, 000 \,001$ mètre).

\myfigure{0.7}{
\tikzinput{couleurs-ex1-01}
} 

  \begin{enumerate}
    \item Quelle couleur a pour longueur d'onde $510$ nanomètres ? Et pour $600$ nanomètres ?
    
    \item Trouve une longueur d'onde possible pour le rouge, le jaune, le violet, le bleu, le bleu ciel.
    
  \end{enumerate}
  
%  \item La perception des couleurs est très variable d'une personne à une autre, selon que l'on regarde un écran ou une feuille imprimée... Dans beaucoup de langues il n'y a qu'un seul mot pour toutes les nuances du bleu au vert. Sur cette palette du vert au bleu, où places-tu la limite ?
%  
%\myfigure{1}{
%\tikzinput{couleurs-ex1-03}
%}   
%  \item \textbf{Illusions d'optique.}
%  
%\myfigure{1}{
%\tikzinput{couleurs-ex1-02}
%} 

%\end{enumerate}

\end{activite}



\begin{activite}[Niveaux de gris]

Une image en \og noir et blanc \fg{} est en fait souvent composée de différents niveaux de gris.

\myfigure{1.2}{
\tikzinput{couleurs-ex2-01}
} 


Il existe plusieurs façons de coder ce niveau de gris :
\begin{itemize}
  \item par le pourcentage de blanc : $0\%$ c'est le noir, $100\%$ c'est le blanc ;
  \item par un nombre réel entre $0$ et $1$ : $0$ c'est noir, $1$ c'est blanc ;
  \item par un nombre entier entre $0$ et $255$ : $0$ c'est noir, $255$ c'est blanc.
\end{itemize}

Voici un exemple de conversion : $25\% = \frac{25}{100} = 0,25$.
Pour la conversion d'une représentation par un réel à une représentation par un entier, on multiplie par $256$ (et pas par $255$ !) et on prend l'entier le plus proche. Par exemple
$0,25 \times 256 = 64$ (sauf $1$ qui devient $255$).

Parmi ces niveaux de gris, retenons uniquement le noir, le blanc et $7$ niveaux intermédiaires.
\myfigure{0.9}{
\footnotesize\tikzinput{couleurs-ex2-02}
}

Colorie le dessin suivant avec le niveau de gris inscrit dans la case.
Toutes les cases sans inscription sont à colorier en gris clair ($0,87$ ou $87 \%$ ou $224$).

\myfigure{0.8}{
\footnotesize\color{black!80}\tikzinput{couleurs-ex2-03}
}

\end{activite}



\begin{activite}[Hexadécimal]

L'écriture hexadécimale est une autre façon de représenter les entiers. Cette écriture utilise $16$ symboles :
$$0 \quad 1 \quad 2 \quad 3 \quad 4 \quad 5 \quad 6 \quad 7 \quad 8 \quad 9 \quad 
A \quad B \quad C \quad D \quad E \quad F$$

Pour différencier l'écriture hexadécimale de l'écriture décimale habituelle, on rajoute en indice \og hex \fg{} à la fin de l'écriture.
Le symbole $A_\hexa$ représente $10$ en écriture décimale, le symbole $B_\hexa$ c'est $11\ldots$ jusqu'au symbole $F_\hexa$ qui représente $15$.

  \begin{center}
    \begin{tabular}[ht]{r|r}
       
      $0$ & $0_\hexa$\\
      $1$ & $1_\hexa$\\
      $2$ & $2_\hexa$\\
      $3$ & $3_\hexa$\\
      $4$ & $4_\hexa$\\
      $5$ & $5_\hexa$\\
      $6$ & $6_\hexa$\\
      $7$ & $7_\hexa$\\

    \end{tabular}\qquad\qquad
    \begin{tabular}[ht]{r|r}
    
      $8$ & $8_\hexa$\\
      $9$  & $9_\hexa$\\
      $10$ & $A_\hexa$\\
      $11$ & $B_\hexa$\\
      $12$ & $C_\hexa$\\
      $13$ & $D_\hexa$\\
      $14$ & $E_\hexa$\\
      $15$ & $F_\hexa$\\    
    
    \end{tabular}
  \end{center}


Nous allons apprendre à écrire tous les nombres de $0$ à $255$ en écriture hexadécimale.
Tu vas voir que deux symboles suffisent !

\begin{enumerate}
  \item \textbf{Hexadécimal vers décimal.}
  
  Pour un nombre écrit avec deux symboles, la formule de conversion de l'écriture hexadécimale en écriture décimale est \myboxinline{$xy_\hexa = 16 \times x +y$}.
  
  
  Exemples :
  \begin{itemize}
    \item $27_\hexa = 16 \times 2 + 7 =  39$,
    
    \item $A3_\hexa = 16 \times 10 + 3 = 163$ (car $A_\hexa$ représente $10$),
    
    \item $2F_\hexa = 16 \times 2 + 15 = 47$ (car $F_\hexa$ représente $15$).
  \end{itemize}
 
 \bigskip
  
  Calcule l'écriture décimale des nombres dont voici l'écriture hexadécimale :
  $$A1_\hexa \quad 2D_\hexa \quad  AC_\hexa \quad  CA_\hexa \quad  B0_\hexa \quad  21_\hexa \quad  
  FF_\hexa \quad  80_\hexa \quad  10_\hexa \quad  AA_\hexa$$ 
  
  \item \textbf{Décimal vers hexadécimal.}
  
  Pour trouver l'écriture hexadécimale d'un entier $n$ compris entre $0$ et $255$, on effectue la division euclidienne de 
  $n$ par $16$ : $n = 16 \times q + r$ avec $0 \le r<16$. L'écriture hexadécimale de $n$ est alors $qr_\hexa$ : le premier symbole est le quotient, le second le reste.
  
   Exemples :
   \begin{itemize}
     \item $n=55$. On divise $55$ par $16$ : le quotient est $3$, le reste est $7$. L'écriture hexadécimale de $55$ est donc $37_\hexa$.
     
     \item $n=44$. On divise $44$ par $16$ : le quotient est $2$, le reste est $12$. L'écriture hexadécimale de $44$ est donc $2C_\hexa$ (car $12$ s'écrit $C_\hexa$.)         
   \end{itemize}   
   
\myfigure{1}{
  \tikzinput{couleurs-ex3}
}  
 
 \bigskip
  Calcule l'écriture hexadécimale des entiers :  
  $$14 \quad 33 \quad 74 \quad 61 \quad 171 \quad 186 \quad 197 \quad 208 \quad 221$$
  
  Calcule et retient l'écriture hexadécimale de $16$, $32$, $64$, $128$, $192$ et $255$.
\end{enumerate}

\end{activite}


\begin{activite}
Le système de couleur RVB décrit une couleur à partir de trois nombres : un pour le niveau de rouge, un pour le niveau de vert et un pour le niveau de bleu.
\`A partir du mélange des trois couleurs rouge, vert et bleu, on obtient les autres couleurs.

\begin{minipage}{0.49\textwidth}
\myfigure{0.7}{
  \tikzinput{couleurs-ex4-01}
}
\end{minipage}
\begin{minipage}{0.49\textwidth}
\myfigure{0.7}{
  \tikzinput{couleurs-ex4-02}
} 
\end{minipage}


Chaque ton de rouge, vert ou bleu sera ici codé par un nombre :
\begin{itemize}
  \item soit un nombre réel entre $0$ et $1$, souvent écrit sous la forme d'un pourcentage,
  \item soit un nombre entier entre $0$ et $255$, qui peut aussi être écrit en hexadécimal par un nombre entre $0_\hexa$ et $FF_\hexa$.
\end{itemize}

Voici les couleurs que l'on obtient lorsque l'on se limite aux niveaux $0\%$, $25\%$,
$50\%$, $75\%$ et $100\%$ (soit $0$, $64$, $128$, $192$ ou $255$, ou encore 
$0_\hexa$, $40_\hexa$, $80_\hexa$, $C0_\hexa$ ou $FF_\hexa$).

\myfigure{0.6}{
\tiny\tikzinput{couleurs-ex4-04}
} 



\begin{enumerate}

  \item Colorie le dessin suivant (le code RVB  est écrit dans chaque case de haut en bas) :
  
\myfigure{0.8}{
\color{black!80}\tikzinput{couleurs-ex4-05}
}  

  \item Complète le tableau suivant: 

\begin{center}
\begin{tabular}{|c|c|c|c|c|}
\hline
Couleur & Nom & Niveau de rouge & Niveau de vert & Niveau de bleu \\ \hline
\cellcolor{red}  & rouge  & 100\% & 0\% & 0\% \\ \hline
\cellcolor{green}& vert  & 0 & 255 & 0 \\ \hline
\cellcolor{blue} & bleu   & $0_\hexa$ & $0_\hexa$ & $FF_\hexa$ \\ \hline
                 & blanc  &  &  &   \\ \hline
                 & noir   &  &  &   \\ \hline
                 & orange &  &  & 0\% \\ \hline  
                 & gris   &  &  &     \\ \hline
                 &        & 255 & 255 & 0 \\ \hline
                 &        & $C0_\hexa$ & $0_\hexa$ & $FF_\hexa$ \\ \hline
                 & rose   &  &  &   \\ \hline
                 &        & 100\% & 100\% & 75\% \\ \hline
\end{tabular}
\end{center} 
  
  \item Si on a $5$ choix de niveau pour le rouge, $5$ choix de niveau pour le vert, $5$ choix de niveau pour le bleu, combien cela fait-il de couleurs possibles ? (Tu peux t'aider des cinq grilles de couleurs ci-dessus.) Si on a maintenant $256$ choix de niveau pour le rouge, pour le vert et pour le bleu, combien cela fait-il de couleurs possibles ?
  
  
  \item Lorsque l'on superpose deux couleurs, on obtient une troisième couleur. La formule est simplement une formule d'addition : $\text{nouveau niveau} = \text{niveau couleur 1} + \text{niveau couleur 2}$.
Par contre, on ne peut pas dépasser la valeur limite de $100\%$ (qui s'écrit aussi $1$ ou $255$ 
ou $FF_\hexa$ selon l'écriture choisie).
La formule exacte est donc (avec des pourcentages) :
\mybox{$\text{nouveau niveau} = \min\big( \text{niveau couleur 1} + \text{niveau couleur 2}, 100\% \Big)$}
 La fonction \og $\min$ \fg{} renvoie le plus petit élément d'une liste : $\min(75,100)= 75$, $\min(125,100)=100$.

Exemple : lorsque l'on ajoute du rouge (code RVB $(100\%,0\%,0\%)$) et du bleu-violet (code RVB $(25\%,0\%,50\%)$)
on obtient :
\begin{itemize}
  \item pour le niveau de rouge : $100\%$ (car si on ajoute $100\%$ et $25\%$, on dépasse $100\%$) ;
  
  \item pour le niveau de vert : $0\%$ (car il n'y a pas de vert dans les deux couleurs) ;
  
  \item pour le niveau de bleu : $50\%$ (c'est $0+50$).
\end{itemize}
Le code RVB de la couleur obtenue est donc $(100\%, 0\%,50\%)$ : c'est du fushia.

\myfigure{0.8}{
\footnotesize\color{black!80}\tikzinput{couleurs-ex4-06}
} 


Complète le tableau suivant dans lequel les couleurs 1 et 2 s'additionnent pour donner une nouvelle couleur :


\begin{center}
\begin{tabular}{|c|c|c|c|c|c|}
\hline
Coul. 1 &  RVB Couleur 1  & Coul. 2 & RVB Couleur 2 & Addition RVB & Couleur \\ \hline
\cellcolor{red}  & $(100\%,0\%,0\%)$ & \cellcolor{green} & $(0\%,100\%,0\%)$ &  & \cellcolor{yellow} \\ \hline
\cellcolor{red}  & $(255,0,0)$       & \cellcolor{blue}   & $(0,0,255)$       &   & \\ \hline
                 & $(0,FF_\hexa,0)$  &                    & $(0,0,FF_\hexa)$   &   & \\ \hline
 & $(25\%,75\%,0\%)$ & & $(50\%,50\%,50\%)$ & & \\ \hline                   
 & $(0,64,0)$ &  & & $(255,192,0)$ & \\ \hline
 &  &  & $(0,64,0)$ & $(128,128,0)$ & \\ \hline     
 & $(0,80_\hexa,C0_\hexa)$ & & $(FF_\hexa,C0_\hexa,40_\hexa)$ & & \\ \hline            
\end{tabular}
\end{center} 
\end{enumerate}

\end{activite}



\end{document}

