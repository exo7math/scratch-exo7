\documentclass[class=report,crop=false, 12pt]{standalone}
\usepackage[screen]{../scratch}

\begin{document}

\titre[P]{Opérations algébriques II}
%===============================



\section*{Objectifs}

\begin{itemize}
  \item \'Ecriture d'un programme par des diagrammes.
  \item Nombres flottants.
\end{itemize}


\section*{Durée}

2 heures (??)

\section*{Les activités}

\begin{itemize}
  \item Les diagrammes pour coder les instructions sont tombés en désuétude mais sont pourtant très pédagogiques ! On respectera les conventions : boîtes rectangulaires pour les instructions, boîtes à bords arrondis pour les entrées/sorties.
  
  \item Ici les programmes sont assez élémentaires, mais c'est pour introduire cette façon de faire pour les feuilles sur les boucles.
  
  \item Nombres flottants : il s'agit ici de comprendre deux choses :
  \begin{itemize}
    \item les ordinateurs ne savent pas stocker les nombres réels, car la mémoire est de taille finie, ce qui se rapproche le plus d'un nombre réel pour un ordinateur, c'est un nombre flottant.
    \item Cela entraîne des "erreurs" : le résultat renvoyé par l'ordinateur n'est pas toujours le résultat attendu par l'être humain ! Mais bien sûr, l'ordinateur ne se trompe pas.

  \item Il y a un autre type d'"erreur" non abordé ici. Les nombres flottants sont stockés en base $2$, ainsi certains nombres qui ont une écriture décimale finie, n'ont pas une écriture finie en base $2$. Par exemple en \texttt{Python} $3 \times 0.1$ n'est pas égal à $0.3$ !
  \end{itemize}
\end{itemize}


\section*{Ressources}


\section*{People}

\begin{itemize}
  \item Auteur : Arnaud Bodin
\end{itemize}


\end{document}


